% Options for packages loaded elsewhere
\PassOptionsToPackage{unicode}{hyperref}
\PassOptionsToPackage{hyphens}{url}
\documentclass[
]{article}
\usepackage{xcolor}
\usepackage[margin=1in]{geometry}
\usepackage{amsmath,amssymb}
\setcounter{secnumdepth}{-\maxdimen} % remove section numbering
\usepackage{iftex}
\ifPDFTeX
  \usepackage[T1]{fontenc}
  \usepackage[utf8]{inputenc}
  \usepackage{textcomp} % provide euro and other symbols
\else % if luatex or xetex
  \usepackage{unicode-math} % this also loads fontspec
  \defaultfontfeatures{Scale=MatchLowercase}
  \defaultfontfeatures[\rmfamily]{Ligatures=TeX,Scale=1}
\fi
\usepackage{lmodern}
\ifPDFTeX\else
  % xetex/luatex font selection
\fi
% Use upquote if available, for straight quotes in verbatim environments
\IfFileExists{upquote.sty}{\usepackage{upquote}}{}
\IfFileExists{microtype.sty}{% use microtype if available
  \usepackage[]{microtype}
  \UseMicrotypeSet[protrusion]{basicmath} % disable protrusion for tt fonts
}{}
\makeatletter
\@ifundefined{KOMAClassName}{% if non-KOMA class
  \IfFileExists{parskip.sty}{%
    \usepackage{parskip}
  }{% else
    \setlength{\parindent}{0pt}
    \setlength{\parskip}{6pt plus 2pt minus 1pt}}
}{% if KOMA class
  \KOMAoptions{parskip=half}}
\makeatother
\usepackage{color}
\usepackage{fancyvrb}
\newcommand{\VerbBar}{|}
\newcommand{\VERB}{\Verb[commandchars=\\\{\}]}
\DefineVerbatimEnvironment{Highlighting}{Verbatim}{commandchars=\\\{\}}
% Add ',fontsize=\small' for more characters per line
\usepackage{framed}
\definecolor{shadecolor}{RGB}{248,248,248}
\newenvironment{Shaded}{\begin{snugshade}}{\end{snugshade}}
\newcommand{\AlertTok}[1]{\textcolor[rgb]{0.94,0.16,0.16}{#1}}
\newcommand{\AnnotationTok}[1]{\textcolor[rgb]{0.56,0.35,0.01}{\textbf{\textit{#1}}}}
\newcommand{\AttributeTok}[1]{\textcolor[rgb]{0.13,0.29,0.53}{#1}}
\newcommand{\BaseNTok}[1]{\textcolor[rgb]{0.00,0.00,0.81}{#1}}
\newcommand{\BuiltInTok}[1]{#1}
\newcommand{\CharTok}[1]{\textcolor[rgb]{0.31,0.60,0.02}{#1}}
\newcommand{\CommentTok}[1]{\textcolor[rgb]{0.56,0.35,0.01}{\textit{#1}}}
\newcommand{\CommentVarTok}[1]{\textcolor[rgb]{0.56,0.35,0.01}{\textbf{\textit{#1}}}}
\newcommand{\ConstantTok}[1]{\textcolor[rgb]{0.56,0.35,0.01}{#1}}
\newcommand{\ControlFlowTok}[1]{\textcolor[rgb]{0.13,0.29,0.53}{\textbf{#1}}}
\newcommand{\DataTypeTok}[1]{\textcolor[rgb]{0.13,0.29,0.53}{#1}}
\newcommand{\DecValTok}[1]{\textcolor[rgb]{0.00,0.00,0.81}{#1}}
\newcommand{\DocumentationTok}[1]{\textcolor[rgb]{0.56,0.35,0.01}{\textbf{\textit{#1}}}}
\newcommand{\ErrorTok}[1]{\textcolor[rgb]{0.64,0.00,0.00}{\textbf{#1}}}
\newcommand{\ExtensionTok}[1]{#1}
\newcommand{\FloatTok}[1]{\textcolor[rgb]{0.00,0.00,0.81}{#1}}
\newcommand{\FunctionTok}[1]{\textcolor[rgb]{0.13,0.29,0.53}{\textbf{#1}}}
\newcommand{\ImportTok}[1]{#1}
\newcommand{\InformationTok}[1]{\textcolor[rgb]{0.56,0.35,0.01}{\textbf{\textit{#1}}}}
\newcommand{\KeywordTok}[1]{\textcolor[rgb]{0.13,0.29,0.53}{\textbf{#1}}}
\newcommand{\NormalTok}[1]{#1}
\newcommand{\OperatorTok}[1]{\textcolor[rgb]{0.81,0.36,0.00}{\textbf{#1}}}
\newcommand{\OtherTok}[1]{\textcolor[rgb]{0.56,0.35,0.01}{#1}}
\newcommand{\PreprocessorTok}[1]{\textcolor[rgb]{0.56,0.35,0.01}{\textit{#1}}}
\newcommand{\RegionMarkerTok}[1]{#1}
\newcommand{\SpecialCharTok}[1]{\textcolor[rgb]{0.81,0.36,0.00}{\textbf{#1}}}
\newcommand{\SpecialStringTok}[1]{\textcolor[rgb]{0.31,0.60,0.02}{#1}}
\newcommand{\StringTok}[1]{\textcolor[rgb]{0.31,0.60,0.02}{#1}}
\newcommand{\VariableTok}[1]{\textcolor[rgb]{0.00,0.00,0.00}{#1}}
\newcommand{\VerbatimStringTok}[1]{\textcolor[rgb]{0.31,0.60,0.02}{#1}}
\newcommand{\WarningTok}[1]{\textcolor[rgb]{0.56,0.35,0.01}{\textbf{\textit{#1}}}}
\usepackage{graphicx}
\makeatletter
\newsavebox\pandoc@box
\newcommand*\pandocbounded[1]{% scales image to fit in text height/width
  \sbox\pandoc@box{#1}%
  \Gscale@div\@tempa{\textheight}{\dimexpr\ht\pandoc@box+\dp\pandoc@box\relax}%
  \Gscale@div\@tempb{\linewidth}{\wd\pandoc@box}%
  \ifdim\@tempb\p@<\@tempa\p@\let\@tempa\@tempb\fi% select the smaller of both
  \ifdim\@tempa\p@<\p@\scalebox{\@tempa}{\usebox\pandoc@box}%
  \else\usebox{\pandoc@box}%
  \fi%
}
% Set default figure placement to htbp
\def\fps@figure{htbp}
\makeatother
\setlength{\emergencystretch}{3em} % prevent overfull lines
\providecommand{\tightlist}{%
  \setlength{\itemsep}{0pt}\setlength{\parskip}{0pt}}
\usepackage{bookmark}
\IfFileExists{xurl.sty}{\usepackage{xurl}}{} % add URL line breaks if available
\urlstyle{same}
\hypersetup{
  pdftitle={CA4},
  pdfauthor={Pavel Gulin, Erling Mysen, Kristian Røhne},
  hidelinks,
  pdfcreator={LaTeX via pandoc}}

\title{CA4}
\author{Pavel Gulin, Erling Mysen, Kristian Røhne}
\date{2025-11-23}

\begin{document}
\maketitle

\begin{Shaded}
\begin{Highlighting}[]
\CommentTok{\#1 a)}
\FunctionTok{library}\NormalTok{(tidyverse)}
\end{Highlighting}
\end{Shaded}

\begin{verbatim}
## -- Attaching core tidyverse packages ------------------------ tidyverse 2.0.0 --
## v dplyr     1.1.4     v readr     2.1.5
## v forcats   1.0.0     v stringr   1.5.2
## v ggplot2   3.5.2     v tibble    3.3.0
## v lubridate 1.9.4     v tidyr     1.3.1
## v purrr     1.1.0     
## -- Conflicts ------------------------------------------ tidyverse_conflicts() --
## x dplyr::filter() masks stats::filter()
## x dplyr::lag()    masks stats::lag()
## i Use the conflicted package (<http://conflicted.r-lib.org/>) to force all conflicts to become errors
\end{verbatim}

\begin{Shaded}
\begin{Highlighting}[]
\CommentTok{\# load both datasets one is the first five keywords (Python, Java, Typescript, Interest Rates, Regulation) worldwide}
\CommentTok{\#Second one is (Volatility, NHL, Ice Skates, Sports Training) \#here ice hockey is seasonal}
\NormalTok{df1 }\OtherTok{\textless{}{-}} \FunctionTok{read\_csv}\NormalTok{(}\StringTok{"multiTimeline.csv"}\NormalTok{, }\AttributeTok{skip =} \DecValTok{2}\NormalTok{) }\CommentTok{\#first two rows are just text from google trends and empty column. }
\end{Highlighting}
\end{Shaded}

\begin{verbatim}
## Rows: 262 Columns: 6
## -- Column specification --------------------------------------------------------
## Delimiter: ","
## dbl  (5): Python: (Worldwide), Java: (Worldwide), TypeScript: (Worldwide), I...
## date (1): Week
## 
## i Use `spec()` to retrieve the full column specification for this data.
## i Specify the column types or set `show_col_types = FALSE` to quiet this message.
\end{verbatim}

\begin{Shaded}
\begin{Highlighting}[]
\NormalTok{df2 }\OtherTok{\textless{}{-}} \FunctionTok{read\_csv}\NormalTok{(}\StringTok{"multiTimeline2.csv"}\NormalTok{, }\AttributeTok{skip =} \DecValTok{2}\NormalTok{) }
\end{Highlighting}
\end{Shaded}

\begin{verbatim}
## Rows: 262 Columns: 5
## -- Column specification --------------------------------------------------------
## Delimiter: ","
## dbl  (4): Volatility: (Worldwide), National Hockey League: (Worldwide), Ice ...
## date (1): Week
## 
## i Use `spec()` to retrieve the full column specification for this data.
## i Specify the column types or set `show_col_types = FALSE` to quiet this message.
\end{verbatim}

\begin{Shaded}
\begin{Highlighting}[]
\NormalTok{df1 }\OtherTok{\textless{}{-}}\NormalTok{ df1 }\SpecialCharTok{\%\textgreater{}\%} \FunctionTok{rename}\NormalTok{(}\AttributeTok{date =} \DecValTok{1}\NormalTok{)}
\NormalTok{df2 }\OtherTok{\textless{}{-}}\NormalTok{ df2 }\SpecialCharTok{\%\textgreater{}\%} \FunctionTok{rename}\NormalTok{(}\AttributeTok{date =} \DecValTok{1}\NormalTok{)}

\CommentTok{\# convert to date}
\NormalTok{df1}\SpecialCharTok{$}\NormalTok{date }\OtherTok{\textless{}{-}} \FunctionTok{as.Date}\NormalTok{(df1}\SpecialCharTok{$}\NormalTok{date)}
\NormalTok{df2}\SpecialCharTok{$}\NormalTok{date }\OtherTok{\textless{}{-}} \FunctionTok{as.Date}\NormalTok{(df2}\SpecialCharTok{$}\NormalTok{date)}

\CommentTok{\# full join to combine all dates}
\NormalTok{df }\OtherTok{\textless{}{-}} \FunctionTok{full\_join}\NormalTok{(df1, df2, }\AttributeTok{by =} \StringTok{"date"}\NormalTok{) }\SpecialCharTok{\%\textgreater{}\%}
  \FunctionTok{arrange}\NormalTok{(date)}
\end{Highlighting}
\end{Shaded}

\begin{Shaded}
\begin{Highlighting}[]
\NormalTok{df\_clean }\OtherTok{\textless{}{-}}\NormalTok{ df }\SpecialCharTok{\%\textgreater{}\%} \CommentTok{\#a lot of wired values in the google dataset}
  \FunctionTok{mutate}\NormalTok{(}\FunctionTok{across}\NormalTok{(}
    \AttributeTok{.cols =} \SpecialCharTok{{-}}\NormalTok{date,}
    \AttributeTok{.fns =} \SpecialCharTok{\textasciitilde{}}\NormalTok{ \{}
\NormalTok{      x }\OtherTok{\textless{}{-}} \FunctionTok{str\_trim}\NormalTok{(.x)}
\NormalTok{      x }\OtherTok{\textless{}{-}} \FunctionTok{str\_replace\_all}\NormalTok{(x, }\StringTok{"\textless{}1"}\NormalTok{, }\StringTok{"0"}\NormalTok{)}
\NormalTok{      x }\OtherTok{\textless{}{-}} \FunctionTok{str\_replace\_all}\NormalTok{(x, }\StringTok{"{-}{-}"}\NormalTok{, }\StringTok{"0"}\NormalTok{)}
\NormalTok{      x }\OtherTok{\textless{}{-}} \FunctionTok{as.numeric}\NormalTok{(x)}
\NormalTok{      x[}\FunctionTok{is.na}\NormalTok{(x)] }\OtherTok{\textless{}{-}} \DecValTok{0}   \CommentTok{\# remove NA }
\NormalTok{      x}
\NormalTok{    \}}
\NormalTok{  ))}

\FunctionTok{names}\NormalTok{(df\_clean) }\OtherTok{\textless{}{-}} \FunctionTok{names}\NormalTok{(df\_clean) }\SpecialCharTok{|\textgreater{}} \CommentTok{\#remove the "(Worldwide)" after each keyword }
  \FunctionTok{str\_replace}\NormalTok{(}\StringTok{": }\SpecialCharTok{\textbackslash{}\textbackslash{}}\StringTok{(Worldwide}\SpecialCharTok{\textbackslash{}\textbackslash{}}\StringTok{)"}\NormalTok{, }\StringTok{""}\NormalTok{) }\SpecialCharTok{|\textgreater{}} 
  \FunctionTok{str\_trim}\NormalTok{()}
\end{Highlighting}
\end{Shaded}

\begin{Shaded}
\begin{Highlighting}[]
\NormalTok{df\_long }\OtherTok{\textless{}{-}}\NormalTok{ df\_clean }\SpecialCharTok{\%\textgreater{}\%}
  \FunctionTok{pivot\_longer}\NormalTok{(}
    \AttributeTok{cols =} \SpecialCharTok{{-}}\NormalTok{date,}
    \AttributeTok{names\_to =} \StringTok{"keyword"}\NormalTok{,}
    \AttributeTok{values\_to =} \StringTok{"value"}
\NormalTok{  )}

\FunctionTok{ggplot}\NormalTok{(df\_long, }\FunctionTok{aes}\NormalTok{(date, value)) }\SpecialCharTok{+}
  \FunctionTok{geom\_line}\NormalTok{(}\AttributeTok{color =} \StringTok{"steelblue"}\NormalTok{) }\SpecialCharTok{+}
  \FunctionTok{facet\_wrap}\NormalTok{(}\SpecialCharTok{\textasciitilde{}}\NormalTok{ keyword, }\AttributeTok{scales =} \StringTok{"free\_y"}\NormalTok{) }\SpecialCharTok{+}
  \FunctionTok{theme\_minimal}\NormalTok{(}\AttributeTok{base\_size =} \DecValTok{14}\NormalTok{) }\SpecialCharTok{+}
  \FunctionTok{labs}\NormalTok{(}
    \AttributeTok{title =} \StringTok{"Google Trends by Keyword"}\NormalTok{,}
    \AttributeTok{x =} \StringTok{"Date"}\NormalTok{,}
    \AttributeTok{y =} \StringTok{"Interest"}
\NormalTok{  )}
\end{Highlighting}
\end{Shaded}

\pandocbounded{\includegraphics[keepaspectratio]{CA4_files/figure-latex/Exercise 1 b}}-1.pdf)
Here we see that the keywords in the same groups somewhat correlate
between each other. Especially the Ice Skate and NHL keywords. Which is
expected as Ice hockey is a seasonal sport. We can also observe that
there is a spike in interest for Regulation, Interest Rate and
Volatility in 2025, which also makes sense as all three keywords are
related to finance. And the spike might have something to do with the
Trumps tariffs and it's impacts on the market. Programming languages
also correlate somewhat, but in a less degree than other keyword.
However we can see that there is a clear downward sloping trend for all
three with a sharp fall in the end of 2025.

\begin{Shaded}
\begin{Highlighting}[]
\FunctionTok{ggplot}\NormalTok{(df\_long, }\FunctionTok{aes}\NormalTok{(keyword, value, }\AttributeTok{fill =}\NormalTok{ keyword)) }\SpecialCharTok{+}
  \FunctionTok{geom\_violin}\NormalTok{(}\AttributeTok{alpha =} \FloatTok{0.6}\NormalTok{) }\SpecialCharTok{+}
  \FunctionTok{geom\_boxplot}\NormalTok{(}\AttributeTok{width =} \FloatTok{0.15}\NormalTok{, }\AttributeTok{color =} \StringTok{"black"}\NormalTok{, }\AttributeTok{alpha =} \FloatTok{0.8}\NormalTok{) }\SpecialCharTok{+}
  \FunctionTok{theme\_minimal}\NormalTok{(}\AttributeTok{base\_size =} \DecValTok{14}\NormalTok{) }\SpecialCharTok{+}
  \FunctionTok{labs}\NormalTok{(}
    \AttributeTok{title =} \StringTok{"Distribution of Google Trends Values"}\NormalTok{,}
    \AttributeTok{x =} \StringTok{""}\NormalTok{,}
    \AttributeTok{y =} \StringTok{"Search Interest"}
\NormalTok{  ) }\SpecialCharTok{+}
  \FunctionTok{theme}\NormalTok{(}\AttributeTok{legend.position =} \StringTok{"none"}\NormalTok{)}
\end{Highlighting}
\end{Shaded}

\pandocbounded{\includegraphics[keepaspectratio]{CA4_files/figure-latex/unnamed-chunk-3-1.pdf}}
The distributions reveal clear differences in search behaviour across
the selected keywords. Python, Java, and the National Hockey League show
the highest overall interest, with large medians and wide interquartile
ranges, reflecting their strong and sustained global popularity. These
variables also display long right-tails, indicating periodic spikes in
attention driven by events such as software news cycles or the ice
hockey season. In contrast, Ice hockey, Interest rate, and Regulation
exhibit moderate medians but greater irregularity, with frequent
mid-sized spikes corresponding to seasonal sports activity or major
economic and policy announcements. Finally, TypeScript, Ice skate, and
Volatility have notably low medians and very narrow distributions,
suggesting niche or highly specialized interest with only occasional
increases in activity. Overall, the distributions separate cleanly into
high-volume stable topics, event-driven mid-range topics, and
consistently low-interest niche topics.

\begin{Shaded}
\begin{Highlighting}[]
\FunctionTok{library}\NormalTok{(forecast)}
\end{Highlighting}
\end{Shaded}

\begin{verbatim}
## Registered S3 method overwritten by 'quantmod':
##   method            from
##   as.zoo.data.frame zoo
\end{verbatim}

\begin{Shaded}
\begin{Highlighting}[]
\FunctionTok{library}\NormalTok{(gridExtra)}
\end{Highlighting}
\end{Shaded}

\begin{verbatim}
## 
## Attaching package: 'gridExtra'
\end{verbatim}

\begin{verbatim}
## The following object is masked from 'package:dplyr':
## 
##     combine
\end{verbatim}

\begin{Shaded}
\begin{Highlighting}[]
\CommentTok{\# 3 years of weekly data}
\NormalTok{lags }\OtherTok{\textless{}{-}} \DecValTok{52} \SpecialCharTok{*} \DecValTok{3}   

\CommentTok{\#ACF plots}
\NormalTok{acf\_plots }\OtherTok{\textless{}{-}}\NormalTok{ df\_long }\SpecialCharTok{\%\textgreater{}\%}
  \FunctionTok{group\_by}\NormalTok{(keyword) }\SpecialCharTok{\%\textgreater{}\%}
  \FunctionTok{summarise}\NormalTok{(}\AttributeTok{ts =} \FunctionTok{list}\NormalTok{(}\FunctionTok{ts}\NormalTok{(value, }\AttributeTok{frequency =} \DecValTok{52}\NormalTok{))) }\SpecialCharTok{\%\textgreater{}\%}
  \FunctionTok{mutate}\NormalTok{(}\AttributeTok{plot =} \FunctionTok{map2}\NormalTok{(ts, keyword, }\SpecialCharTok{\textasciitilde{}}\NormalTok{\{}
    \FunctionTok{ggAcf}\NormalTok{(.x, }\AttributeTok{lag.max =}\NormalTok{ lags) }\SpecialCharTok{+} 
      \FunctionTok{ggtitle}\NormalTok{(}\FunctionTok{paste}\NormalTok{(}\StringTok{"ACF {-}"}\NormalTok{, .y)) }\SpecialCharTok{+}
      \FunctionTok{theme\_minimal}\NormalTok{()}
\NormalTok{  \}))}

\FunctionTok{do.call}\NormalTok{(}\StringTok{"grid.arrange"}\NormalTok{, }\FunctionTok{c}\NormalTok{(acf\_plots}\SpecialCharTok{$}\NormalTok{plot, }\AttributeTok{ncol =} \DecValTok{3}\NormalTok{))}
\end{Highlighting}
\end{Shaded}

\pandocbounded{\includegraphics[keepaspectratio]{CA4_files/figure-latex/Exercise 1 c}}-1.pdf)
The ACF plots show three distinct behavioral patterns across the nine
time series. Ice hockey, Ice skate, and National Hockey League all
exhibit strong, repeating peaks at regular intervals (around 52, 104,
and 156 weeks), indicating clear annual seasonality driven by the winter
sports cycle. Python, Java, and to a weaker extent TypeScript display
persistent autocorrelation that decays slowly, consistent with long-term
stable interest in programming languages and gradual trends rather than
sharp seasonal effects. In contrast, Interest rate, Regulation, and
Volatility show rapidly decaying autocorrelation, suggesting short-lived
shocks tied to economic or policy events with limited long-term
persistence. Overall, the ACFs cleanly separate the series into highly
seasonal (sports), trend-persistent (programming), and event-driven
(financial regulation) categories.

\begin{Shaded}
\begin{Highlighting}[]
\NormalTok{pacf\_plots }\OtherTok{\textless{}{-}}\NormalTok{ df\_long }\SpecialCharTok{\%\textgreater{}\%}
  \FunctionTok{group\_by}\NormalTok{(keyword) }\SpecialCharTok{\%\textgreater{}\%}
  \FunctionTok{summarise}\NormalTok{(}\AttributeTok{ts =} \FunctionTok{list}\NormalTok{(}\FunctionTok{ts}\NormalTok{(value, }\AttributeTok{frequency =} \DecValTok{52}\NormalTok{))) }\SpecialCharTok{\%\textgreater{}\%}
  \FunctionTok{mutate}\NormalTok{(}\AttributeTok{plot =} \FunctionTok{map2}\NormalTok{(ts, keyword, }\SpecialCharTok{\textasciitilde{}}\NormalTok{\{}
    \FunctionTok{ggPacf}\NormalTok{(.x, }\AttributeTok{lag.max =}\NormalTok{ lags) }\SpecialCharTok{+} 
      \FunctionTok{ggtitle}\NormalTok{(}\FunctionTok{paste}\NormalTok{(}\StringTok{"PACF {-}"}\NormalTok{, .y)) }\SpecialCharTok{+}
      \FunctionTok{theme\_minimal}\NormalTok{()}
\NormalTok{  \}))}

\FunctionTok{do.call}\NormalTok{(}\StringTok{"grid.arrange"}\NormalTok{, }\FunctionTok{c}\NormalTok{(pacf\_plots}\SpecialCharTok{$}\NormalTok{plot, }\AttributeTok{ncol =} \DecValTok{3}\NormalTok{))}
\end{Highlighting}
\end{Shaded}

\pandocbounded{\includegraphics[keepaspectratio]{CA4_files/figure-latex/Exercise 1c}}-1.pdf)
Across all nine series, the PACF plots show a dominant lag-1 spike,
indicating that each time series is mainly driven by short-term
dependence. Beyond lag 1, partial autocorrelations drop quickly toward
zero for every keyword. The sports-related series (Ice hockey, Ice
skate, NHL) show only weak seasonal signals at longer lags, confirming
that their strong annual patterns in the ACF come from smooth
seasonality rather than higher-order AR structure. The programming
languages (Python, Java, TypeScript) similarly behave like low-order AR
processes with persistence driven mostly by trends. The financial terms
(Interest rate, Regulation, Volatility) show almost no significant
partial autocorrelation past lag 1, reflecting short-lived, event-driven
shocks. Overall, the PACF indicates minimal higher-order autoregressive
structure across all groups.

\begin{Shaded}
\begin{Highlighting}[]
\FunctionTok{library}\NormalTok{(dtw)}
\end{Highlighting}
\end{Shaded}

\begin{verbatim}
## Loading required package: proxy
\end{verbatim}

\begin{verbatim}
## 
## Attaching package: 'proxy'
\end{verbatim}

\begin{verbatim}
## The following objects are masked from 'package:stats':
## 
##     as.dist, dist
\end{verbatim}

\begin{verbatim}
## The following object is masked from 'package:base':
## 
##     as.matrix
\end{verbatim}

\begin{verbatim}
## Loaded dtw v1.23-1. See ?dtw for help, citation("dtw") for use in publication.
\end{verbatim}

\begin{Shaded}
\begin{Highlighting}[]
\FunctionTok{library}\NormalTok{(pheatmap)}

\NormalTok{ts\_mat }\OtherTok{\textless{}{-}}\NormalTok{ df\_clean }\SpecialCharTok{\%\textgreater{}\%}
  \FunctionTok{select}\NormalTok{(}\SpecialCharTok{{-}}\NormalTok{date) }\SpecialCharTok{\%\textgreater{}\%}
  \FunctionTok{as.matrix}\NormalTok{()}

\NormalTok{ts\_mat }\OtherTok{\textless{}{-}}\NormalTok{ df\_clean }\SpecialCharTok{\%\textgreater{}\%} \FunctionTok{select}\NormalTok{(}\SpecialCharTok{{-}}\NormalTok{date) }\SpecialCharTok{\%\textgreater{}\%} \FunctionTok{as.matrix}\NormalTok{() }\CommentTok{\#simple point{-}wise difference, ignores temporal shifts.}
\NormalTok{dist\_euclid }\OtherTok{\textless{}{-}} \FunctionTok{dist}\NormalTok{(}\FunctionTok{t}\NormalTok{(ts\_mat), }\AttributeTok{method =} \StringTok{"euclidean"}\NormalTok{)}

\NormalTok{dist\_corr }\OtherTok{\textless{}{-}} \FunctionTok{as.dist}\NormalTok{(}\DecValTok{1} \SpecialCharTok{{-}} \FunctionTok{cor}\NormalTok{(ts\_mat)) }\CommentTok{\#distance correlation. Captures similarity in shape regardless of scale.}

\NormalTok{keywords }\OtherTok{\textless{}{-}} \FunctionTok{colnames}\NormalTok{(ts\_mat)}
\NormalTok{dist\_dtw }\OtherTok{\textless{}{-}} \FunctionTok{matrix}\NormalTok{(}\DecValTok{0}\NormalTok{, }\FunctionTok{ncol}\NormalTok{(ts\_mat), }\FunctionTok{ncol}\NormalTok{(ts\_mat))}
\FunctionTok{rownames}\NormalTok{(dist\_dtw) }\OtherTok{\textless{}{-}} \FunctionTok{colnames}\NormalTok{(dist\_dtw) }\OtherTok{\textless{}{-}}\NormalTok{ keywords}

\ControlFlowTok{for}\NormalTok{(i }\ControlFlowTok{in} \DecValTok{1}\SpecialCharTok{:}\FunctionTok{ncol}\NormalTok{(ts\_mat))\{}
  \ControlFlowTok{for}\NormalTok{(j }\ControlFlowTok{in} \DecValTok{1}\SpecialCharTok{:}\FunctionTok{ncol}\NormalTok{(ts\_mat))\{}
\NormalTok{    dist\_dtw[i,j] }\OtherTok{\textless{}{-}} \FunctionTok{dtw}\NormalTok{(ts\_mat[,i], ts\_mat[,j])}\SpecialCharTok{$}\NormalTok{distance}
\NormalTok{  \}}
\NormalTok{\}}

\NormalTok{dist\_dtw }\OtherTok{\textless{}{-}} \FunctionTok{as.dist}\NormalTok{(dist\_dtw) }\CommentTok{\#dynamic time warping distance}

\NormalTok{dist\_arima }\OtherTok{\textless{}{-}} \FunctionTok{matrix}\NormalTok{(}\DecValTok{0}\NormalTok{, }\FunctionTok{ncol}\NormalTok{(ts\_mat), }\FunctionTok{ncol}\NormalTok{(ts\_mat)) }\CommentTok{\#ARIMA{-}based dissimilarity}
\FunctionTok{rownames}\NormalTok{(dist\_arima) }\OtherTok{\textless{}{-}} \FunctionTok{colnames}\NormalTok{(dist\_arima) }\OtherTok{\textless{}{-}} \FunctionTok{colnames}\NormalTok{(ts\_mat)}

\ControlFlowTok{for}\NormalTok{(i }\ControlFlowTok{in} \DecValTok{1}\SpecialCharTok{:}\FunctionTok{ncol}\NormalTok{(ts\_mat))\{}
  \ControlFlowTok{for}\NormalTok{(j }\ControlFlowTok{in} \DecValTok{1}\SpecialCharTok{:}\FunctionTok{ncol}\NormalTok{(ts\_mat))\{}
\NormalTok{    fit1 }\OtherTok{\textless{}{-}} \FunctionTok{auto.arima}\NormalTok{(ts\_mat[,i])}
\NormalTok{    fit2 }\OtherTok{\textless{}{-}} \FunctionTok{auto.arima}\NormalTok{(ts\_mat[,j])}
\NormalTok{    dist\_arima[i,j] }\OtherTok{\textless{}{-}} \FunctionTok{abs}\NormalTok{(fit1}\SpecialCharTok{$}\NormalTok{aic }\SpecialCharTok{{-}}\NormalTok{ fit2}\SpecialCharTok{$}\NormalTok{aic)}
\NormalTok{  \}}
\NormalTok{\}}

\NormalTok{dist\_arima }\OtherTok{\textless{}{-}} \FunctionTok{as.dist}\NormalTok{(dist\_arima)}
\end{Highlighting}
\end{Shaded}

\begin{Shaded}
\begin{Highlighting}[]
\FunctionTok{pheatmap}\NormalTok{(}\FunctionTok{as.matrix}\NormalTok{(dist\_euclid), }\AttributeTok{main =} \StringTok{"Euclidean Distance"}\NormalTok{)}
\end{Highlighting}
\end{Shaded}

\pandocbounded{\includegraphics[keepaspectratio]{CA4_files/figure-latex/unnamed-chunk-4-1.pdf}}

\begin{Shaded}
\begin{Highlighting}[]
\FunctionTok{pheatmap}\NormalTok{(}\FunctionTok{as.matrix}\NormalTok{(dist\_corr), }\AttributeTok{main =} \StringTok{"Correlation Distance"}\NormalTok{)}
\end{Highlighting}
\end{Shaded}

\pandocbounded{\includegraphics[keepaspectratio]{CA4_files/figure-latex/unnamed-chunk-5-1.pdf}}

\begin{Shaded}
\begin{Highlighting}[]
\FunctionTok{pheatmap}\NormalTok{(}\FunctionTok{as.matrix}\NormalTok{(dist\_dtw), }\AttributeTok{main =} \StringTok{"DTW Distance"}\NormalTok{)}
\end{Highlighting}
\end{Shaded}

\pandocbounded{\includegraphics[keepaspectratio]{CA4_files/figure-latex/unnamed-chunk-6-1.pdf}}

\begin{Shaded}
\begin{Highlighting}[]
\FunctionTok{pheatmap}\NormalTok{(}\FunctionTok{as.matrix}\NormalTok{(dist\_arima), }\AttributeTok{main =} \StringTok{"ARIMA Model Distance"}\NormalTok{)}
\end{Highlighting}
\end{Shaded}

\pandocbounded{\includegraphics[keepaspectratio]{CA4_files/figure-latex/unnamed-chunk-7-1.pdf}}
All four distance measures reveal the same broad grouping of the series.
The sports-related keywords (Ice hockey, Ice skate, NHL) consistently
cluster together, especially under correlation and DTW, reflecting their
strong shared seasonality. The programming languages (Python, Java,
TypeScript) also form a tight group across all measures due to their
smooth, trend-like behavior. The financial terms (Interest rate,
Regulation, Volatility) appear closer to each other under Euclidean and
correlation distance but are more separated under DTW and ARIMA
distance, indicating that while they share low overall magnitude, their
spikes and temporal dynamics differ. Overall, the measures consistently
separate the data into seasonal, trend-driven, and event-driven groups.

\begin{Shaded}
\begin{Highlighting}[]
\FunctionTok{library}\NormalTok{(ggplot2)}
\FunctionTok{library}\NormalTok{(ggdendro)}
\FunctionTok{library}\NormalTok{(gridExtra)}

\NormalTok{plot\_dend }\OtherTok{\textless{}{-}} \ControlFlowTok{function}\NormalTok{(dist\_matrix, title)\{}
\NormalTok{  hc }\OtherTok{\textless{}{-}} \FunctionTok{hclust}\NormalTok{(dist\_matrix, }\AttributeTok{method =} \StringTok{"complete"}\NormalTok{)}
  \FunctionTok{ggdendrogram}\NormalTok{(hc, }\AttributeTok{rotate =} \ConstantTok{TRUE}\NormalTok{, }\AttributeTok{size =} \DecValTok{2}\NormalTok{) }\SpecialCharTok{+}
    \FunctionTok{ggtitle}\NormalTok{(title) }\SpecialCharTok{+}
    \FunctionTok{theme\_minimal}\NormalTok{()}
\NormalTok{\}}

\NormalTok{p1 }\OtherTok{\textless{}{-}} \FunctionTok{plot\_dend}\NormalTok{(dist\_euclid, }\StringTok{"Euclidean Distance"}\NormalTok{)}
\NormalTok{p2 }\OtherTok{\textless{}{-}} \FunctionTok{plot\_dend}\NormalTok{(dist\_corr, }\StringTok{"Correlation Distance"}\NormalTok{)}
\NormalTok{p3 }\OtherTok{\textless{}{-}} \FunctionTok{plot\_dend}\NormalTok{(dist\_dtw, }\StringTok{"DTW Distance"}\NormalTok{)}
\NormalTok{p4 }\OtherTok{\textless{}{-}} \FunctionTok{plot\_dend}\NormalTok{(dist\_arima, }\StringTok{"ARIMA Model Distance"}\NormalTok{)}

\FunctionTok{grid.arrange}\NormalTok{(p1, p2, p3, p4, }\AttributeTok{ncol =} \DecValTok{2}\NormalTok{)}
\end{Highlighting}
\end{Shaded}

\pandocbounded{\includegraphics[keepaspectratio]{CA4_files/figure-latex/Exercise 2a}}-1.pdf)
The four dendograms shows that the clustering structure is highly
sensetive to the distance metric choice. The only measure that
clulstrize the topics correctly is actually the correleation
distance.That might be because correlation captures how similarly two
time series moves together. And obviously keywords in the same category
move somewhat similarly. The euclidian distance captures magnitude
differences. So python and java is high volume searches here, that are
separated from mid and low volume searches. (Ice Hockey, NHL,
Regulation, Interest Rate) and (Ice skate, Volatility, TypeScript)
respectively. DTW distance forms clusters based on shape similarity
regardless of time alignment, which explains why series with similar
fluctuation patterns (even if their peaks dont align) are grouped
together. The ARIMA-based distance groups time series according to the
underlying model parameters that best describe them (trend, seasonality,
persistence, and autocorrelation). Unlike Euclidean or correlation
distance, this metric does not compare the raw series directly, but
instead compares the statistical structure captured by each fitted ARIMA
model. As a result, the clusters reflect similarities in the dynamics of
the series rather than their magnitude or point-wise shape.

\begin{Shaded}
\begin{Highlighting}[]
\FunctionTok{library}\NormalTok{(mclust)}
\end{Highlighting}
\end{Shaded}

\begin{verbatim}
## Package 'mclust' version 6.1.2
## Type 'citation("mclust")' for citing this R package in publications.
\end{verbatim}

\begin{verbatim}
## 
## Attaching package: 'mclust'
\end{verbatim}

\begin{verbatim}
## The following object is masked from 'package:dplyr':
## 
##     count
\end{verbatim}

\begin{verbatim}
## The following object is masked from 'package:purrr':
## 
##     map
\end{verbatim}

\begin{Shaded}
\begin{Highlighting}[]
\CommentTok{\# Perform hierarchical clustering}
\NormalTok{hc\_euclid }\OtherTok{\textless{}{-}} \FunctionTok{hclust}\NormalTok{(dist\_euclid, }\AttributeTok{method =} \StringTok{"complete"}\NormalTok{)}
\NormalTok{hc\_corr   }\OtherTok{\textless{}{-}} \FunctionTok{hclust}\NormalTok{(dist\_corr,   }\AttributeTok{method =} \StringTok{"complete"}\NormalTok{)}
\NormalTok{hc\_dtw    }\OtherTok{\textless{}{-}} \FunctionTok{hclust}\NormalTok{(dist\_dtw,    }\AttributeTok{method =} \StringTok{"complete"}\NormalTok{)}
\NormalTok{hc\_arima  }\OtherTok{\textless{}{-}} \FunctionTok{hclust}\NormalTok{(dist\_arima,  }\AttributeTok{method =} \StringTok{"complete"}\NormalTok{)}

\CommentTok{\# Cut into 3 clusters (sports, programming, finance)}
\NormalTok{pred\_euclid }\OtherTok{\textless{}{-}} \FunctionTok{cutree}\NormalTok{(hc\_euclid, }\AttributeTok{k =} \DecValTok{3}\NormalTok{)}
\NormalTok{pred\_corr   }\OtherTok{\textless{}{-}} \FunctionTok{cutree}\NormalTok{(hc\_corr,   }\AttributeTok{k =} \DecValTok{3}\NormalTok{)}
\NormalTok{pred\_dtw    }\OtherTok{\textless{}{-}} \FunctionTok{cutree}\NormalTok{(hc\_dtw,    }\AttributeTok{k =} \DecValTok{3}\NormalTok{)}
\NormalTok{pred\_arima  }\OtherTok{\textless{}{-}} \FunctionTok{cutree}\NormalTok{(hc\_arima,  }\AttributeTok{k =} \DecValTok{3}\NormalTok{)}

\CommentTok{\# True labels (make sure order matches your columns)}
\NormalTok{true\_labels }\OtherTok{\textless{}{-}} \FunctionTok{c}\NormalTok{(}
  \StringTok{"sports"}\NormalTok{, }\StringTok{"sports"}\NormalTok{, }\StringTok{"sports"}\NormalTok{,       }\CommentTok{\# Ice hockey, Ice skate, NHL}
  \StringTok{"programming"}\NormalTok{, }\StringTok{"programming"}\NormalTok{, }\StringTok{"programming"}\NormalTok{,   }\CommentTok{\# Python, Java, TypeScript}
  \StringTok{"finance"}\NormalTok{, }\StringTok{"finance"}\NormalTok{, }\StringTok{"finance"}     \CommentTok{\# Interest rate, Regulation, Volatility}
\NormalTok{)}

\CommentTok{\# Compute ARI}
\NormalTok{ari\_euclid }\OtherTok{\textless{}{-}} \FunctionTok{adjustedRandIndex}\NormalTok{(pred\_euclid, true\_labels)}
\NormalTok{ari\_corr   }\OtherTok{\textless{}{-}} \FunctionTok{adjustedRandIndex}\NormalTok{(pred\_corr,   true\_labels)}
\NormalTok{ari\_dtw    }\OtherTok{\textless{}{-}} \FunctionTok{adjustedRandIndex}\NormalTok{(pred\_dtw,    true\_labels)}
\NormalTok{ari\_arima  }\OtherTok{\textless{}{-}} \FunctionTok{adjustedRandIndex}\NormalTok{(pred\_arima,  true\_labels)}

\NormalTok{ari\_euclid; ari\_corr; ari\_dtw; ari\_arima}
\end{Highlighting}
\end{Shaded}

\begin{verbatim}
## [1] 0.07142857
\end{verbatim}

\begin{verbatim}
## [1] 1
\end{verbatim}

\begin{verbatim}
## [1] -0.1612903
\end{verbatim}

\begin{verbatim}
## [1] -0.03703704
\end{verbatim}

Correlation distance: ARI = 1.00 This is perfect agreement. Correlation
distance is the only method that correctly identifies all three
clusters. This makes sense because it groups series by co-movement
patterns, which align strongly with topic-driven behavior (e.g., ice
hockey seasonality, tech trend stability, macro-finance co-movements).

Euclidean distance: ARI ≈ 0.07 Very close to random. Euclidean distance
is dominated by absolute magnitude, not pattern, causing unrelated
series with similar scale to cluster together. It completely fails to
capture the thematic structure of the data.

DTW distance: ARI ≈ --0.16 Worse than random. DTW over-aligns local
fluctuations, causing series with totally different semantic meaning to
be matched simply because their wiggles look similar when warped. This
leads to nonsensical clusters from a topical perspective.

ARIMA model distance: ARI ≈ --0.04 Also worse than random. ARIMA-based
distance groups series by model dynamics (trend, seasonality,
autocorrelation), not by topic. Because the Google Trends series vary
widely in noise level and structure, the resulting clusters do not
reflect the true categories.

\begin{Shaded}
\begin{Highlighting}[]
\CommentTok{\# correlation distance}
\NormalTok{dist\_corr }\OtherTok{\textless{}{-}} \FunctionTok{as.dist}\NormalTok{(}\DecValTok{1} \SpecialCharTok{{-}} \FunctionTok{cor}\NormalTok{(ts\_mat))}

\CommentTok{\# hierarchical clustering}
\NormalTok{hc\_corr }\OtherTok{\textless{}{-}} \FunctionTok{hclust}\NormalTok{(dist\_corr, }\AttributeTok{method =} \StringTok{"complete"}\NormalTok{)}

\CommentTok{\# cut into 3 clusters}
\NormalTok{cluster\_assignments }\OtherTok{\textless{}{-}} \FunctionTok{cutree}\NormalTok{(hc\_corr, }\AttributeTok{k =} \DecValTok{3}\NormalTok{)}

\NormalTok{cluster\_assignments}
\end{Highlighting}
\end{Shaded}

\begin{verbatim}
##                 Python                   Java             TypeScript 
##                      1                      1                      1 
##          Interest rate             Regulation             Volatility 
##                      2                      2                      2 
## National Hockey League              Ice skate             Ice hockey 
##                      3                      3                      3
\end{verbatim}

\begin{Shaded}
\begin{Highlighting}[]
\NormalTok{cluster\_df }\OtherTok{\textless{}{-}} \FunctionTok{data.frame}\NormalTok{(}
  \AttributeTok{keyword =} \FunctionTok{colnames}\NormalTok{(ts\_mat),}
  \AttributeTok{cluster =} \FunctionTok{factor}\NormalTok{(cluster\_assignments)}
\NormalTok{)}

\NormalTok{df\_plot }\OtherTok{\textless{}{-}}\NormalTok{ df\_long }\SpecialCharTok{\%\textgreater{}\%}
  \FunctionTok{left\_join}\NormalTok{(cluster\_df, }\AttributeTok{by =} \StringTok{"keyword"}\NormalTok{)}
\end{Highlighting}
\end{Shaded}

\begin{Shaded}
\begin{Highlighting}[]
\FunctionTok{ggplot}\NormalTok{(df\_plot, }\FunctionTok{aes}\NormalTok{(}\AttributeTok{x =}\NormalTok{ date, }\AttributeTok{y =}\NormalTok{ value, }\AttributeTok{color =}\NormalTok{ cluster)) }\SpecialCharTok{+}
  \FunctionTok{geom\_line}\NormalTok{() }\SpecialCharTok{+}
  \FunctionTok{facet\_wrap}\NormalTok{(}\SpecialCharTok{\textasciitilde{}}\NormalTok{ keyword, }\AttributeTok{scales =} \StringTok{"free\_y"}\NormalTok{, }\AttributeTok{ncol =} \DecValTok{3}\NormalTok{) }\SpecialCharTok{+}
  \FunctionTok{theme\_minimal}\NormalTok{(}\AttributeTok{base\_size =} \DecValTok{13}\NormalTok{) }\SpecialCharTok{+}
  \FunctionTok{labs}\NormalTok{(}
    \AttributeTok{title =} \StringTok{"Time Series Clustered Using Correlation Distance"}\NormalTok{,}
    \AttributeTok{x =} \StringTok{""}\NormalTok{,}
    \AttributeTok{y =} \StringTok{"Search Interest"}
\NormalTok{  )}
\end{Highlighting}
\end{Shaded}

\pandocbounded{\includegraphics[keepaspectratio]{CA4_files/figure-latex/unnamed-chunk-9-1.pdf}}
The correlation-based clustering cleanly separates the series into the
expected three groups. The sports series (Ice hockey, Ice skate, NHL)
cluster together due to their strong shared seasonality, while the
programming languages (Python, Java, TypeScript) form a second group
with similar long-term trends. The financial terms (Interest rate,
Regulation, Volatility) cluster together as event-driven series with
irregular spikes. TypeScript is the only borderline case because its
low, flat pattern is less similar to Python and Java, but it is not
clearly misclassified. Overall, the clustering matches the true
structure very well.

Exercise 3 a) The Markov property means the process has no memory: once
you know the current state anything that happened before becomes useless
for predicting. Formally, the conditional distribution of the next state
depends only on the present state, not the whole history.''Given the
present, the past is irrelevant for predicting the future.''

A simple Markov process can be imagined as a small board game with three
squares: A, B, and C. At each step you flip a coin to decide whether you
move left or right, and if you're at an edge you simply bounce back
because there's nowhere further to go. Your next position depends only
on your current square and the new coin flip, not on any of the previous
moves you made. This makes the process Markov: the future is determined
entirely by the present, with no memory of the past.

The stock market is not a Markov process because future price movements
depend on far more than just the current price. Past trends, volatility
regimes, market sentiment, macro announcements, liquidity shocks, and
structural breaks all influence how prices evolve, and this information
is not contained in the current price alone. Even if you know today's
price, you are missing critical context like momentum, volatility
clustering, and order flow dynamics. Because predicting tomorrow
requires more than just the present state, the stock market violates the
Markov property.

Exercise 3 b) The transition matrix A tells you the probabilities of
jumping from one state to another in a single time step. Each row
corresponds to the current state, and each entry is the probability that
the chain moves from state i to j state next period.

The state distribution vector π(t) tells you the probabilities of being
in each state at time t.

Exercise 3 c) The steady state vector π∗ represents the long-run
distribution of the Markov chain: the probabilities of being in each
state after the system has run for a long time and settled into
equilibrium.

Property 1 means π∗ is unchanged by the transition dynamics. If the
chain is already in distribution π∗, one more step does nothing. It's a
fixed point of the Markov evolution. This ensures stationarity.

Property 2 means the probabilities sums to 1. This ensures probability
distribution.

Property 3 means that each component must be non-negative, because
probabilities can't be negative.

Exercise 3 d) Emission matrix \(B\) tells you how the hidden state
\(X_t\) produces the observable output \(Y_t\). Each entry \[
B_{ij} = \Pr(Y_t = j \mid X_t = i)
\] is the probability that the system emits observation \(j\) when it is
in hidden state \(i\). In other words, \(A\) governs how the hidden
state evolves, while \(B\) governs how the hidden state reveals itself
to you.

Marginal distribution vector gives the overall probabilities of
observing each possible value of \(Y_t\) at time \(t\), after accounting
for all the ways you might get those observations from the hidden
states.

\begin{Shaded}
\begin{Highlighting}[]
\FunctionTok{library}\NormalTok{(HiddenMarkov)}
\FunctionTok{library}\NormalTok{(tidyverse)}
\FunctionTok{library}\NormalTok{(HiddenMarkov)}
\FunctionTok{library}\NormalTok{(patchwork)}

\NormalTok{kw1 }\OtherTok{\textless{}{-}} \StringTok{"National Hockey League"}
\NormalTok{kw2 }\OtherTok{\textless{}{-}} \StringTok{"Interest rate"}
\NormalTok{kw3 }\OtherTok{\textless{}{-}} \StringTok{"Python"}

\CommentTok{\#fit + plot one hmm}
\NormalTok{fit\_and\_plot\_hmm }\OtherTok{\textless{}{-}} \ControlFlowTok{function}\NormalTok{(df\_k) \{}
\NormalTok{  df\_k }\OtherTok{\textless{}{-}}\NormalTok{ df\_k }\SpecialCharTok{\%\textgreater{}\%} \FunctionTok{arrange}\NormalTok{(date)}
\NormalTok{  x }\OtherTok{\textless{}{-}}\NormalTok{ df\_k}\SpecialCharTok{$}\NormalTok{value}
  
  \CommentTok{\# Build model}
\NormalTok{  model }\OtherTok{\textless{}{-}} \FunctionTok{dthmm}\NormalTok{(}
    \AttributeTok{x      =}\NormalTok{ x,}
    \AttributeTok{Pi     =} \FunctionTok{matrix}\NormalTok{(}\FunctionTok{c}\NormalTok{(}\FloatTok{0.9}\NormalTok{,}\FloatTok{0.1}\NormalTok{,}
                      \FloatTok{0.1}\NormalTok{,}\FloatTok{0.9}\NormalTok{), }\AttributeTok{nrow =} \DecValTok{2}\NormalTok{),}
    \AttributeTok{delta  =} \FunctionTok{c}\NormalTok{(}\FloatTok{0.5}\NormalTok{, }\FloatTok{0.5}\NormalTok{),}
    \AttributeTok{distn  =} \StringTok{"norm"}\NormalTok{,}
    \AttributeTok{pm     =} \FunctionTok{list}\NormalTok{(}
      \AttributeTok{mean =} \FunctionTok{c}\NormalTok{(}\FunctionTok{mean}\NormalTok{(x) }\SpecialCharTok{{-}} \FunctionTok{sd}\NormalTok{(x), }\FunctionTok{mean}\NormalTok{(x) }\SpecialCharTok{+} \FunctionTok{sd}\NormalTok{(x)),}
      \AttributeTok{sd   =} \FunctionTok{c}\NormalTok{(}\FunctionTok{sd}\NormalTok{(x), }\FunctionTok{sd}\NormalTok{(x))}
\NormalTok{    )}
\NormalTok{  )}
  
  \CommentTok{\# Fit with Baum–Welch}
\NormalTok{    tmp }\OtherTok{\textless{}{-}} \FunctionTok{capture.output}\NormalTok{( }\CommentTok{\#remove the noisy console output for better pdf readability}
\NormalTok{      fit }\OtherTok{\textless{}{-}} \FunctionTok{BaumWelch}\NormalTok{(model)}
\NormalTok{  )}

  
  \CommentTok{\# Decode most likely states}
\NormalTok{  states }\OtherTok{\textless{}{-}} \FunctionTok{Viterbi}\NormalTok{(fit)}
\NormalTok{  df\_k}\SpecialCharTok{$}\NormalTok{state }\OtherTok{\textless{}{-}} \FunctionTok{factor}\NormalTok{(states)}
  
  \CommentTok{\# Plot}
  \FunctionTok{ggplot}\NormalTok{(df\_k, }\FunctionTok{aes}\NormalTok{(date, value, }\AttributeTok{color =}\NormalTok{ state)) }\SpecialCharTok{+}
    \FunctionTok{geom\_line}\NormalTok{(}\AttributeTok{size =} \DecValTok{1}\NormalTok{) }\SpecialCharTok{+}
    \FunctionTok{theme\_minimal}\NormalTok{() }\SpecialCharTok{+}
    \FunctionTok{labs}\NormalTok{(}
      \AttributeTok{title =} \FunctionTok{paste}\NormalTok{(}\StringTok{"HMM (2 states) for"}\NormalTok{, df\_k}\SpecialCharTok{$}\NormalTok{keyword[}\DecValTok{1}\NormalTok{]),}
      \AttributeTok{x =} \StringTok{"Date"}\NormalTok{,}
      \AttributeTok{y =} \StringTok{"Value"}\NormalTok{,}
      \AttributeTok{color =} \StringTok{"State"}
\NormalTok{    )}
\NormalTok{\}}


\CommentTok{\#fit and plot}
\NormalTok{df\_NHL }\OtherTok{\textless{}{-}}\NormalTok{ df\_long }\SpecialCharTok{\%\textgreater{}\%} \FunctionTok{filter}\NormalTok{(keyword }\SpecialCharTok{==}\NormalTok{ kw1)}
\NormalTok{p1 }\OtherTok{\textless{}{-}} \FunctionTok{fit\_and\_plot\_hmm}\NormalTok{(df\_NHL)}
\end{Highlighting}
\end{Shaded}

\begin{verbatim}
## Warning: Using `size` aesthetic for lines was deprecated in ggplot2 3.4.0.
## i Please use `linewidth` instead.
## This warning is displayed once every 8 hours.
## Call `lifecycle::last_lifecycle_warnings()` to see where this warning was
## generated.
\end{verbatim}

\begin{Shaded}
\begin{Highlighting}[]
\NormalTok{df\_IR }\OtherTok{\textless{}{-}}\NormalTok{ df\_long }\SpecialCharTok{\%\textgreater{}\%} \FunctionTok{filter}\NormalTok{(keyword }\SpecialCharTok{==}\NormalTok{ kw2)}
\NormalTok{p2 }\OtherTok{\textless{}{-}} \FunctionTok{fit\_and\_plot\_hmm}\NormalTok{(df\_IR)}

\NormalTok{df\_Python }\OtherTok{\textless{}{-}}\NormalTok{ df\_long }\SpecialCharTok{\%\textgreater{}\%} \FunctionTok{filter}\NormalTok{(keyword }\SpecialCharTok{==}\NormalTok{ kw3)}
\NormalTok{p3 }\OtherTok{\textless{}{-}} \FunctionTok{fit\_and\_plot\_hmm}\NormalTok{(df\_Python)}

\CommentTok{\# Show plots}
\NormalTok{combined }\OtherTok{\textless{}{-}}\NormalTok{ p1 }\SpecialCharTok{/}\NormalTok{ p2 }\SpecialCharTok{/}\NormalTok{ p3}
\NormalTok{combined}
\end{Highlighting}
\end{Shaded}

\pandocbounded{\includegraphics[keepaspectratio]{CA4_files/figure-latex/Exercise 3 e}}-1.pdf)
The two--state HMM divides each time series into a ``low-activity'' and
a ``high-activity'' regime, based purely on the level and variability of
the observations. Because the model is constrained to use two states, it
will always produce such a segmentation, so the regimes should be
interpreted as simple statistical clusters rather than deep structural
patterns. For the purposes of this assignment, however, the results
appropriately illustrate how HMMs identify and visualize changes in
underlying behavior over time.

Exercise 4a) Using Euler's identity,
\(e^{i\omega t} = \cos(\omega t) + i\sin(\omega t)\), so the real part
is cos(ωt), the imaginary part is sin(ωt), and the magnitude is
\(|e^{i\omega t}| = 1\).

For r\textgreater0, the expression
\(r e^{i\omega t} = r\cos(\omega t) + i\,r\sin(\omega t)\) represents a
rotating vector of constant length r in the complex plane. Its real and
imaginary components are rcos(ωt) and rsin(ωt), and its magnitude is
\(|r e^{i\omega t}| = r\)

Exercise 4 b) X(ω) describes how the signal's energy is distributed
across frequencies. If X(ω) is large around ω=0, then x(t) contains a
strong low-frequency or constant component, meaning it varies slowly or
has a large average value.

Exercise 4 c) \#\#\# Determining the Frequencies from the Fourier
Transform

The signal is given by\\
\[
x(t) = \sin(\omega_1 t) + 2\sin(\omega_2 t) + 4\sin(\omega_3 t),
\] and the magnitude of its Fourier transform \(|X(\omega)|\) is shown
in the figure.

The Fourier transform of a sum of sinusoids produces peaks at the
corresponding positive and negative frequencies. In the plot of
\(|X(\omega)|\), we observe clear peaks at\\
\[
\omega \approx 2,\qquad \omega \approx 4,\qquad \omega \approx 7,
\] and the relative peak heights match the amplitudes of the components
in \(x(t)\):\\
the smallest peak corresponds to amplitude \(1\), the intermediate peak
to amplitude \(2\), and the tallest peak to amplitude \(4\).

Therefore, the correct frequencies are\\
\[
\omega_1 = 2,\qquad \omega_2 = 4,\qquad \omega_3 = 7.
\]

\begin{Shaded}
\begin{Highlighting}[]
\FunctionTok{library}\NormalTok{(tidyverse)}

\CommentTok{\#function to compute FFT magnitude }
\NormalTok{compute\_fft }\OtherTok{\textless{}{-}} \ControlFlowTok{function}\NormalTok{(x) \{}
\NormalTok{  N }\OtherTok{\textless{}{-}} \FunctionTok{length}\NormalTok{(x)}
\NormalTok{  F }\OtherTok{\textless{}{-}} \FunctionTok{fft}\NormalTok{(x)                }\CommentTok{\# complex FFT result}
\NormalTok{  mag }\OtherTok{\textless{}{-}} \FunctionTok{Mod}\NormalTok{(F) }\SpecialCharTok{/}\NormalTok{ N          }\CommentTok{\# magnitude spectrum}
  \FunctionTok{tibble}\NormalTok{(}\AttributeTok{freq =} \DecValTok{0}\SpecialCharTok{:}\NormalTok{(N}\DecValTok{{-}1}\NormalTok{), }\AttributeTok{mag =}\NormalTok{ mag)}
\NormalTok{\}}

\CommentTok{\# Compute FFT for each time series}
\NormalTok{fft\_nhl }\OtherTok{\textless{}{-}} \FunctionTok{compute\_fft}\NormalTok{(df\_NHL}\SpecialCharTok{$}\NormalTok{value)}
\NormalTok{fft\_ir  }\OtherTok{\textless{}{-}} \FunctionTok{compute\_fft}\NormalTok{(df\_IR}\SpecialCharTok{$}\NormalTok{value)}
\NormalTok{fft\_py  }\OtherTok{\textless{}{-}} \FunctionTok{compute\_fft}\NormalTok{(df\_IR}\SpecialCharTok{$}\NormalTok{value)}

\CommentTok{\# {-}{-}{-} Plot function {-}{-}{-}}
\NormalTok{plot\_fft }\OtherTok{\textless{}{-}} \ControlFlowTok{function}\NormalTok{(df\_fft, title\_label) \{}
  \FunctionTok{ggplot}\NormalTok{(df\_fft, }\FunctionTok{aes}\NormalTok{(freq, mag)) }\SpecialCharTok{+}
    \FunctionTok{geom\_line}\NormalTok{(}\AttributeTok{color =} \StringTok{"steelblue"}\NormalTok{, }\AttributeTok{linewidth =} \FloatTok{0.8}\NormalTok{) }\SpecialCharTok{+}
    \FunctionTok{theme\_minimal}\NormalTok{() }\SpecialCharTok{+}
    \FunctionTok{labs}\NormalTok{(}
      \AttributeTok{title =} \FunctionTok{paste}\NormalTok{(}\StringTok{"FFT Magnitude for"}\NormalTok{, title\_label),}
      \AttributeTok{x =} \StringTok{"Frequency Index"}\NormalTok{,}
      \AttributeTok{y =} \StringTok{"|X(ω)|"}
\NormalTok{    )}
\NormalTok{\}}

\CommentTok{\# Produce the plots}
\NormalTok{p\_fft\_nhl }\OtherTok{\textless{}{-}} \FunctionTok{plot\_fft}\NormalTok{(fft\_nhl, }\StringTok{"NHL"}\NormalTok{)}
\NormalTok{p\_fft\_ir  }\OtherTok{\textless{}{-}} \FunctionTok{plot\_fft}\NormalTok{(fft\_ir, }\StringTok{"Interest rate"}\NormalTok{)}
\NormalTok{p\_fft\_py  }\OtherTok{\textless{}{-}} \FunctionTok{plot\_fft}\NormalTok{(fft\_py, }\StringTok{"Python"}\NormalTok{)}

\NormalTok{(p\_fft\_nhl }\SpecialCharTok{/}\NormalTok{ p\_fft\_ir }\SpecialCharTok{/}\NormalTok{ p\_fft\_py)}
\end{Highlighting}
\end{Shaded}

\pandocbounded{\includegraphics[keepaspectratio]{CA4_files/figure-latex/Exercise 4d}}-1.pdf)
The FFT magnitude spectra show a dominant peak at frequency index 0 for
all three time series, indicating a strong DC component and slow overall
variation. The ``Interest rate'' and ``Python'' series exhibit flat,
low-magnitude spectra beyond the DC peak, suggesting they contain no
strong repeating or periodic structure. In contrast, the ``NHL'' series
shows additional low-frequency peaks, reflecting weak periodic patterns
consistent with seasonal or event-driven fluctuations.

\end{document}
